\documentclass[12pt]{article}
\usepackage[utf8]{inputenc}
\usepackage{amsmath,amssymb, fourier}
\usepackage{siunitx}
\usepackage{booktabs}
\usepackage{wasysym}
\usepackage[hidelinks]{hyperref}
\usepackage{graphicx}
\usepackage{xcolor}
\usepackage[a4paper,top=3cm, bottom=3cm, margin=2.5cm]{geometry}

\newcommand{\aggiunto}[1]{\textcolor{blue}{#1}}
\newcommand{\diff}{\text{d}}
\newcommand{\difft}[1]{\ensuremath{\frac{\diff #1}{\diff t}}}

\title{Personal notes}

\begin{document}
\maketitle


\section{Equivalent for meteors of Ragazzoni 2020}
\subsection{Geometry/visibility}
A meteor has an height $h$ above the point $Q$ on the (spherical) surface of the Earth.
\begin{figure}[h]
	\centering
	\includegraphics[width=5cm]{photo_2022-03-07_18-39-35}
\end{figure}
The angular separation between $Q$ and the observing site $P$ is $\theta$. $O$ is the center of the Earth, $M$ the position of the meteor and $M'$ its projection over the vertical of $P$. We have
\begin{equation}
\begin{split}
	&MM' = (R+h)\sin\theta\\
	&PM' =(R+h)\cos\theta-R
\end{split}
\end{equation}
between the observed angular zenit distance $z$ and the other parameters is
\begin{equation}
	\tan z = \frac{(R+h)\sin\theta}{(R+h)\cos\theta -R}
\end{equation}
To be meaningful, the denominator at the right hand side must be positive, i.e.\
\begin{equation}
	\cos\theta \ge \frac{R}{R+h}
\end{equation}
which set the visibility limit in terms of phase angle between the meteor and the observed, wrt the Earth.
Note when $\cos\theta\to R/(R+h)$ then $\tan z\to \pi/2$ i.e.\ the meteor is observed at the horizon.
%
\paragraph{Low altitude approximation.} The formula above becomes
\begin{equation}
	\tan z \approx \frac{\sin\theta}{\cos\theta -1} = \tan\left(\frac{\theta}{2}\right)\rightarrow\ z\approx \theta/2
\end{equation}
in fact if $(R+h)\approx R$ we can act as if earth surface and the spherical surface at the height of the meteor were the same one. Then the $z$ is a circumference angle while $\theta$ is a center angle insisting on the same arc $\widearc{PQ}$ so they will be one half of the other.
The visibility limit becomes
\begin{equation}
	\cos\theta\ge \left(1+\frac{h}{R}\right)^{-1}\approx 1-\frac{h}{R}
\end{equation}
Ablation typically starts at $80-\SI{90}{km}$ [Ceplecha+98] so assuming $(h+R)\approx R$ is legit at first approximation.

\section{Motion of meteors}

\subsection{Intrinsic motion}
\paragraph{Shape of the trajectory.} Before entering the atmosphere the motion of a meteoroid consist on its intrinsic solar orbital motion + acceleration due to Earth attraction. Typical velocity ranges from \SI{11.2}{km/s} (pure Earth attraction) to \SI{72.8}{km/s} (solar +  Earth attraction).\\

\noindent
During the ablation the trajectory is a straight line, because we neglect
\begin{itemize}
	\item the effect of gravitational attraction of the Earth, which is much smaller than the air drag exerted by the atmosphere;
	\item the effect of the curvature of the planet. This is because during the burn meteors covers hundreds of km $\ll R$.
\end{itemize}

\noindent
In general the motion along this straight line is not uniform as the meteoroid is being slew down by air drag.

\paragraph{Equation of motion.} Motion on the trajectory is described by the set of differential equation [Cepleca+98]
\begin{equation}
	\begin{split}
	&a = \difft{v}=-\Gamma A \rho_m^{-2/3}\rho m^{-1/3} v^2\\
	&\dot m = \difft{m}=-\frac{\Lambda A}{2\xi}\rho_m^{-2/3} m^{2/3} v^3
	\end{split}
\end{equation}
where
\begin{itemize}
	\item $\Gamma$ is the drag coefficient (fraction of momentum transferred to the air from the body, ranges from 0 for no exchange of momentum, to 2 for a perfectly elastic impact),
	\item $\Lambda$ the heat transfer coefficient (fraction of kinetic energy converted into heat),
	\item $\xi$ the energy for the ablation of a unit mass,
	\item $A = S m^{-2/3}\rho_m^{2/3}=S/V^{2/3}$ the shape factor ($S$ the cross-section, for cube $A=1$, for a sphere $A\approx1.2$, getting more elongated toward the direction of motion makes $A$ grow),
	\item $\rho_m$ the bulk density,
	\item $m$ the mass and
	\item $v$ the velocity.
\end{itemize}

\noindent
Usually independent parameters are gathered in the two terms [Cepleca+98]
\begin{equation}
	\begin{split}
	&\sigma =\frac{\Lambda}{2\xi \Gamma}\\
	&K = \Lambda A\rho_m^{-2/3}
	\end{split}
\end{equation}
respectively the ablation and the shape-density coefficients. Motion equation become [adapted from Cepleca+98]
\begin{equation}
	\begin{split}
	&a = -K\rho m^{-1/3} v^2\\
	&\dot m = -K\sigma m^{2/3} v^3
	\end{split}
\end{equation}

\paragraph{Dependence on the height.}
Note that $\rho=\rho(h)=\rho[h(t)]$, i.e.\ we need a further relation to express the variation of atmospheric density as a function of the height. A good approximation of the atmospheric density profile is
\begin{equation}
	\rho = \rho_0 e^{-h/H}
\end{equation}
where the scale height $H$ is
\begin{equation}
	H=\frac{P_0}{g\rho_0}\sim \SI{8}{km}
\end{equation}
with $P$ the pressure. The pedix represents the value at the sea level, for $h=0$.\\

\noindent
We introduce the third equation of motion
\begin{equation}
	v_h =\difft{h}=\cos(z_0) v
\end{equation}
where $z_0$ is the zenit angle of the meteor radiant (i.e.\ inclination of the trajector wrt the local vertical).\\

\paragraph{Final equations.}
The final set of ODE that solve the motion of a meteor is
\begin{equation}
	\begin{split}
	&a = -K\rho_0 e^{-h/H} m^{-1/3} v^2\\
	&\dot m = -K\sigma m^{2/3} v^3\\
	&v_h = \cos z_0 v
	\end{split}
\end{equation}
with $K$ and $\sigma$ that comes from the properties and composition of the body, $z_0$ from the pre-atmospheric motion while $\rho_0$ and $H$ are two constants dependent on the atmospheric model.\\

\noindent
Note we assumed, more or less implicitly:
\begin{itemize}
	\item straight-line trajectory (see above);
	\item constant inclination $z_0$ (direction of the vertical does not changes significantly below the short meteor path);
	\item $\sigma$ and $K$ constant. It is assumed that the shape and properties does not change significantly during ablation.
\end{itemize}

\paragraph{Solution.} An example for $\rho_m=\SI{3000}{kg/\centi\meter\cubed}$, $\Lambda=1$, $A=1$, $\sigma=\SI{0.03}{s^2/km^2}$ and $z_0=\ang{30}$. Initial condition velocity \SI{25}{km/s}, mass \SI{1.0}{kg} and \SI{150}{km} of altitude. Integration over \SI{10}{s}, \num{300} time-steps.
\begin{figure}[h]
	\centering
	\includegraphics[width=0.7\linewidth]{./index}
\end{figure}

\subsection{Apparent motion}
Due to the short path and low altitude approximation, the motion can be naturally expressed using a rectangular coordinate system in the neighborhood of the meteor, i.e.\ assuming locally the Earth to be flat right below the meteor.\\

\noindent
An object at height $h$ can actually be observed up to a distance $\approx\sqrt{2Rh}$ (in low altitude approximation) from its projection on the ground. In the case of meteors with height up to a hundred of km, they can be observed up to \SI{e3}{km}, which is no longer negligible wrt the curvature.\\

\noindent
If $\theta$ is the angular separation between the vertical of the meteor and the observer the apparent inclination of the meteor is $z=z_0-\theta$

\section{Meteor background -- simplified}

\paragraph{Procedure.} A tentative approach.
\begin{figure}[h]
	\centering
	\includegraphics[width=0.5\linewidth]{photo_2022-03-10_12-04-43}
\end{figure}
\begin{enumerate}
	\item For [Cook+78] the meteor flux through the atmosphere respect
	\begin{equation}
		\log\Phi = -a+bM = -17.89+0.534 M
	\end{equation}
	where $\Phi$ is the number flux (meteor/cm$^2$/s) with an absolute magnitude lower (i.e.\ brighter) than $M$.
	
	
	\item Assume all the meteors burn in a shell of thickness $\Delta h$ at average height $h$.
	
	\item From a point $P$ is observed at direction $z$, wrt the vertical, through the shell. The volume of the shell column of unit area $\diff A$ observed from $P$ is
	\begin{equation}
		\diff V =\diff A\diff\rho \approx\diff A\frac{\Delta h}{\cos z}
	\end{equation}
	
	\item We pass from the cumulative flux $\Phi$ to the differential count $\varphi$ (number of meteor per surface and time unit with a magnitude range between $M$ and $M+\diff M$)
	\begin{equation}
		\varphi(M) =\frac{\diff \Phi}{\diff M}= \frac{\diff}{\diff M}10^{-a+bM} = 10^{-a} \ln 10 b 10^{bM}
	\end{equation}
	The flux per volume unit $\psi$ will have the same expression but different units.
	
	\item The total number of object of magnitude $M$ seen under a unit area $\diff A$ from $P$ is
	\begin{equation}
		\diff N=\psi(M)\diff V = \psi \diff A\frac{\Delta h}{\cos z}
	\end{equation}
	
	\item The intensity of a single object with magnitude $M$ is
	\begin{equation}
		I = 10^{-0.4M}
	\end{equation}
	so the total intensity from the volume $\diff V$ is
	\begin{equation}
		I_\text{tot} = \int_{M_\text{min}}^{M_\text{max}}\diff N I\diff M = \diff A\frac{\Delta h}{\cos z} \frac{b}{b-0.4} 10^{-a+(b-0.4)M_\text{max}}
	\end{equation}
	we can set $M_\text{min}\to -\infty$ and have no problems with the boundaries since $b>0.4$ and the exponential will tend to 0, when evaluated at $-\infty$.
	
	\item According to [Blaauw+16] the relation between absolute magnitude and mass is
	\begin{equation}
		M = -8.75\log(v_\infty)-2.25\log m+11.59
	\end{equation}
	with velocity in m/s and mass in grams. In particular the maximum magnitude (faintest meteor) is the one with smallest mass and lowest pre-atmospheric velocity.
	
	A conservative approach may consist on considering $v_{\infty,\text{min}}=\SI{11.2}{km/s}$ (escape velocity) and as $m_\text{min}$ the one associated to the smallest meteoroids.
	Conventionally the boundary between dust and meteoroid is for a size of $\SI{30}{\micro\metre}=\SI{3e-5}{m}$, the boundary under which the body is small enough to dissipate the heat due to atmospheric friction without vaporizing. The equivalent volume, assuming a rocky density of $\SI{3}{g/cm\cubed}$ is
	\begin{equation}
		m_\text{min}\approx (\num{3e-5})^3 \cdot 3000 \sim \SI{e-7}{g}
	\end{equation}
	
	The final value for the maximum magnitude is (conservatively) $M_\text{max}=\SI{18.37}{mag}$.
	
	\item The total magnitude observed from the column will be
	\begin{equation}
		M_\text{tot} = -2.5\log I_\text{tot}
	\end{equation}
			
	\item An observer at \SI{100}{km} from the column will see $M=m$ by definition. At this distance it will observe the surface area $\diff A$ under the solid angle
	\begin{equation}
		\diff\Omega \cong \frac{\diff A}{\rho^2} 206265^2 = \num{7.037e6}\, \diff A
	\end{equation}
	with $\diff A$ in km$^2$ units and $\diff \Omega$ in \si{arcsec\square}.
	
	\item By definition the surface brightness at 100\,km is
	\begin{equation}
		S = M +2.5\log \Omega
	\end{equation}
	But since both the intensity and the solid angle scale with $1/\rho^2$ at the end $S$ will be independent on the distance, i.e.\ we can apply compute $S$ once.
	
	In our case
	\begin{equation}
		S = -2.5\log\Delta h+2.5\log\cos z+2.5\log\left(\frac{b}{b-0.4}\right)+2.5a-2.5(b-0.4)M_\text{max}-4.69
	\end{equation}
	
\end{enumerate}

\paragraph{Result.}
If we assume $a$ and $b$ from [Cook+78] a thickness of the shell of $\Delta h =\SI{20}{km}$ and $M_\text{max}$ as above we have $S$ as a function of the zenit angle $z$
\begin{equation}
	S(z)= 37.82+2.5\log(\cos z)
\end{equation}
\begin{figure}[h]
	\centering
	\includegraphics[width=0.7\linewidth]{S_z}
\end{figure}

\end{document}