\documentclass[12pt]{article}
\usepackage[utf8]{inputenc}
\usepackage{amsmath,amssymb, fourier}
\usepackage{siunitx}
\usepackage{booktabs}
\usepackage{wasysym}
\usepackage[hidelinks]{hyperref}
\usepackage{graphicx}
\usepackage{xcolor}
\usepackage[a4paper,top=3cm, bottom=3cm, margin=2.5cm]{geometry}

\newcommand{\aggiunto}[1]{\textcolor{blue}{#1}}
\newcommand{\diff}{\text{d}}
\newcommand{\difft}[1]{\ensuremath{\frac{\diff #1}{\diff t}}}

\title{Personal notes}

\begin{document}
\maketitle

\aggiunto{
\section{Derivations of Ragazzoni 2020}
%
\subsection{Geometry}
Satellite orbital plane has inclination $\phi_\text{max}$ wrt equator, i.e.\ $\phi_\text{max}$ is the highest latitude on the surface where the satellite passes at the zenit.
\begin{figure}[h]
	\centering
	\includegraphics[width=0.8\linewidth]{"Schermata da 2022-03-07 13-07-32"}
\end{figure}
%
\noindent
The height above the surface is $h$ and the Earth radius $R$.\\
%
\noindent
The surface wrapped by the satellite around the earth is a spherical zone $\text{area} = 2\pi\times \text{radius}\times\text{height}$ of area
\begin{equation}
	S = 2\pi (R+h)\times 2(R+h)\sin\phi_\text{max}=4\pi (R+h)^2 \sin\phi_\text{max}
\end{equation}
%
\noindent
\paragraph{Satellites above the horizon.} An observer in a point $P$ well inside the area wrapped by the satellite will see a limited of the total surface, i.e.\ the spherical cusp given by the intersection of the satellite are with the horizon plane in $P$
\begin{equation}
	S_1 = 2\pi (R+h)h
\end{equation}
%
\noindent
For a population of satellites that share the same $\phi_\text{max}$ an $h$ the fraction of satellites above the horizon is
\begin{equation}
	\eta_1 =\frac{S_1}{S} = \frac{1}{2\sin\phi_\text{max}}\frac{h}{R+h}
\end{equation}
In the limit of low-altitude orbit $h\ll R$ and
\begin{equation}
	\eta_1\approx \frac{1}{2\sin\phi_\text{max}}\frac{h}{R}
\end{equation}
%
\noindent
\paragraph{Satellites close to the zenit.} In the typical case we are interested on satellite with a zenital angle $z<z_\text{max}$. The relative fraction of satellite close to the zenit in low altitude regime is only
\begin{equation}
	\eta_2 \approx \frac{1}{2\sin\phi_\text{max}}\frac{h^2}{R^2}\tan^2 z_\text{max}
\end{equation}
i.e.\ drastically reduced.
%
\paragraph{Satellite visibility.} To be visible a satellite must be illuminated by the Sun. Relation between $h$ and the satellite-earth-sun phase angle $\Psi$ is
\begin{equation}
	\cos\Psi = \frac{R}{R+h}
\end{equation}
}

\section{Equivalent for meteors}
\subsection{Geometry/visibility}
A meteor has an height $h$ above the point $Q$ on the (spherical) surface of the Earth.
\begin{figure}[h]
	\centering
	\includegraphics[width=5cm]{photo_2022-03-07_18-39-35}
\end{figure}
The angular separation between $Q$ and the observing site $P$ is $\theta$. $O$ is the center of the Earth, $M$ the position of the meteor and $M'$ its projection over the vertical of $P$. We have
\begin{equation}
\begin{split}
	&MM' = (R+h)\sin\theta\\
	&PM' =(R+h)\cos\theta-R
\end{split}
\end{equation}
between the observed angular zenit distance $z$ and the other parameters is
\begin{equation}
	\tan z = \frac{(R+h)\sin\theta}{(R+h)\cos\theta -R}
\end{equation}
To be meaningful, the denominator at the right hand side must be positive, i.e.\
\begin{equation}
	\cos\theta \ge \frac{R}{R+h}
\end{equation}
which set the visibility limit in terms of phase angle between the meteor and the observed, wrt the Earth.
Note when $\cos\theta\to R/(R+h)$ then $\tan z\to \pi/2$ i.e.\ the meteor is observed at the horizon.
%
\paragraph{Low altitude approximation.} The formula above becomes
\begin{equation}
	\tan z \approx \frac{\sin\theta}{\cos\theta -1} = \tan\left(\frac{\theta}{2}\right)\rightarrow\ z\approx \theta/2
\end{equation}
in fact if $(R+h)\approx R$ we can act as if earth surface and the spherical surface at the height of the meteor were the same one. Then the $z$ is a circumference angle while $\theta$ is a center angle insisting on the same arc $\widearc{PQ}$ so they will be one half of the other.
The visibility limit becomes
\begin{equation}
	\cos\theta\ge \left(1+\frac{h}{R}\right)^{-1}\approx 1-\frac{h}{R}
\end{equation}
Ablation typically starts at $80-\SI{90}{km}$ [Ceplecha+98] so assuming $(h+R)\approx R$ is legit at first approximation.

\subsection{Intrinsic motion}
\paragraph{Shape of the trajectory.} Before entering the atmosphere the motion of a meteoroid consist on its intrinsic solar orbital motion + acceleration due to Earth attraction. Typical velocity ranges from \SI{11.2}{km/s} (pure Earth attraction) to \SI{72.8}{km/s} (solar +  Earth attraction).\\

\noindent
During the ablation the trajectory is a straight line, because we neglect
\begin{itemize}
	\item the effect of gravitational attraction of the Earth, which is much smaller than the air drag exerted by the atmosphere;
	\item the effect of the curvature of the planet. This is because during the burn meteors covers hundreds of km $\ll R$.
\end{itemize}

\noindent
In general the motion along this straight line is not uniform as the meteoroid is being slew down by air drag.

\paragraph{Equation of motion.} Motion on the trajectory is described by the set of differential equation [Cepleca+98]
\begin{equation}
	\begin{split}
	&a = \difft{v}=-\Gamma A \rho_m^{-2/3}\rho m^{-1/3} v^2\\
	&\dot m = \difft{m}=-\frac{\Lambda A}{2\xi}\rho_m^{-2/3} m^{2/3} v^3
	\end{split}
\end{equation}
where
\begin{itemize}
	\item $\Gamma$ is the drag coefficient (fraction of momentum transferred to the air from the body, ranges from 0 for no exchange of momentum, to 2 for a perfectly elastic impact),
	\item $\Lambda$ the heat transfer coefficient (fraction of kinetic energy converted into heat),
	\item $\xi$ the energy for the ablation of a unit mass,
	\item $A = S m^{-2/3}\rho_m^{2/3}=S/V^{2/3}$ the shape factor ($S$ the cross-section, for cube $A=1$, for a sphere $A\approx1.2$, getting more elongated toward the direction of motion makes $A$ grow),
	\item $\rho_m$ the bulk density,
	\item $m$ the mass and
	\item $v$ the velocity.
\end{itemize}

\noindent
Usually independent parameters are gathered in the two terms [Cepleca+98]
\begin{equation}
	\begin{split}
	&\sigma =\frac{\Lambda}{2\xi \Gamma}\\
	&K = \Lambda A\rho_m^{-2/3}
	\end{split}
\end{equation}
respectively the ablation and the shape-density coefficients. Motion equation become [adapted from Cepleca+98]
\begin{equation}
	\begin{split}
	&a = -K\rho m^{-1/3} v^2\\
	&\dot m = -K\sigma m^{2/3} v^3
	\end{split}
\end{equation}

\paragraph{Dependence on the height.}
Note that $\rho=\rho(h)=\rho[h(t)]$, i.e.\ we need a further relation to express the variation of atmospheric density as a function of the height. A good approximation of the atmospheric density profile is
\begin{equation}
	\rho = \rho_0 e^{-h/H}
\end{equation}
where the scale height $H$ is
\begin{equation}
	H=\frac{P_0}{g\rho_0}\sim \SI{8}{km}
\end{equation}
with $P$ the pressure. The pedix represents the value at the sea level, for $h=0$.\\

\noindent
We introduce the third equation of motion
\begin{equation}
	v_h =\difft{h}=\cos(z_0) v
\end{equation}
where $z_0$ is the zenit angle of the meteor radiant (i.e.\ inclination of the trajector wrt the local vertical).\\

\paragraph{Final equations.}
The final set of ODE that solve the motion of a meteor is
\begin{equation}
	\begin{split}
	&a = -K\rho_0 e^{-h/H} m^{-1/3} v^2\\
	&\dot m = -K\sigma m^{2/3} v^3\\
	&v_h = \cos z_0 v
	\end{split}
\end{equation}
with $K$ and $\sigma$ that comes from the properties and composition of the body, $z_0$ from the pre-atmospheric motion while $\rho_0$ and $H$ are two constants dependent on the atmospheric model.\\

\noindent
Note we assumed, more or less implicitly:
\begin{itemize}
	\item straight-line trajectory (see above);
	\item constant inclination $z_0$ (direction of the vertical does not changes significantly below the short meteor path);
	\item $\sigma$ and $K$ constant. It is assumed that the shape and properties does not change significantly during ablation.
\end{itemize}

\paragraph{Solution.} An example for $\rho_m=\SI{3000}{kg/\centi\meter\cubed}$, $\Lambda=1$, $A=1$, $\sigma=\SI{0.03}{s^2/km^2}$ and $z_0=\ang{30}$. Initial condition velocity \SI{25}{km/s}, mass \SI{1.0}{kg} and \SI{150}{km} of altitude. Integration over \SI{10}{s}, \num{300} time-steps.
\begin{figure}[h]
	\centering
	\includegraphics[width=0.7\linewidth]{./index}
\end{figure}

\subsection{Apparent motion}
Due to the short path and low altitude approximation, the motion can be naturally expressed using a rectangular coordinate system in the neighborhood of the meteor, i.e.\ assuming locally the Earth to be flat right below the meteor.\\

\noindent
An object at height $h$ can actually be observed up to a distance $\approx\sqrt{2Rh}$ (in low altitude approximation) from its projection on the ground. In the case of meteors with height up to a hundred of km, they can be observed up to \SI{e3}{km}, which is no longer negligible wrt the curvature.\\

\noindent
If $\theta$ is the angular separation between the vertical of the meteor and the observer the apparent inclination of the meteor is $z=z_0-\theta$

\section{Meteor background -- simplified}
Assumptions:
\begin{itemize}
	\item Constant isotropic trajectories for any observer on the Earth.
	\item Cumulative density flux $\phi$ (meteors/area/s) as a function of the absolute magnitude (100\, km) $M$
	\begin{equation}
		\log\phi = -a+b M
	\end{equation}
	with $a$ and $b$ positive constant. From [Cook+78].
	\item Cumulative density flux $\phi_m$ as a function of the initial mass $m$
	\begin{equation}
	\log\phi_m = -a_m-b_m \log m
	\end{equation}
	with $a_m$ and $b_m$ positive constant. From [Cook+78].
\end{itemize}

\paragraph{Procedure.} A tentative approach
\begin{enumerate}
	\item Choose the limit altitude $h_\text{max}$ where meteors can no longer occur (too thin atmosphere).
	\item Compute the range of visibility of an object at $h_\text{max}$. In LAA it is simply
	\begin{equation}
		d_\text{max} \approx\sqrt{2Rh_\text{max}}
	\end{equation}
	\item Populate the circular area of radius $d_\text{max}$ with a number of meteoroids as a function of the mass provided by $\phi_m$. Inverting the log-log relation
	\begin{equation}
		\phi_m = A m^{b_m}
	\end{equation}
	with $A=e^{-a_m}$ a new constant.
	\item Assign to each object with a given mass an initial velocity and a random orientation (due to isotropy). Initial velocity can be sampled from a uniform distribution in the range 11.2-72.8\,km/s for example.
	\item Integrate the motion of each particle starting from $h_\text{max}$.
	\item Project the motion into the observer spherical reference frame.
\end{enumerate}

\paragraph{Alternative (statistical).} Very wide area and (\num{6e6} cells) each with thousands of object require statistical treatment.
\begin{enumerate}
	\item For [Cook+78] the meteor flux through the atmosphere respect
	\begin{equation}
		\log\phi = -17.89+0.534 M
	\end{equation}
	where $\Phi$ is the number flux (meteor/cm$^2$/s) with an absolute magnitude lower (i.e.\ brighter) than $M$.
	\item Assume a reference height $h^*$ at which we assume all the ablation occurs, i.e.\ as if all the meteor background comes from a spherical shell around the Earth at height $h^*$ from the surface.
	\item Consider a surface unit on the shall at zenit distance $z$ from seen from the observing point. The distance from the observer is
	\begin{equation}
		\rho = -R\cos z+\sqrt{(R+h^*)^2-R^2\sin^2 z}\approx\frac{h^*}{\cos z}
	\end{equation}
	\item The relation between intrinsic 100\,km magnitude and the apparent one is
	\begin{equation}
		m = M +5\log\left(\frac{\rho}{\SI{100}{km}}\right) = M -10 + 5 \log\left(h^*\right)-5\log(\cos z)
	\end{equation}
	with $h^*$ in km.
	\item Flux of meteor with magnitude $M$ (within an infinitesimal bin of width $\diff M$) are
	\begin{equation}
		\varphi (M)=\frac{\diff \Phi}{\diff M}=b\ln 10\Phi(M)
	\end{equation}
	\item We can pass from magnitudes to intensities with
	\begin{equation}
		I = 10^{-0.4M}
	\end{equation}
	Total intensity per area unit per time unit will be the sum over all meteor of all all magnitudes:
	\begin{equation}
		I_\text{tot}=\int_{-\infty}^{+\infty}\varphi(M)I(M)\diff M =  10^{-a}\frac{b}{b-0.4}\left[10^{(b-0.4)M}\right]_{M_\text{min}}^{M_\text{max}}
	\end{equation}
	we can set $M_\text{min}\to -\infty$ and have no problems with the boundaries since $b>0.4$ and the exponential will tend to 0, when evaluated at $-\infty$.
	\item We pass again to total magnitudes computing the total absolute magnitude from from a surface and time unit.
	\begin{equation}
		M_\text{tot} = -2.5\log I_\text{tot}=2.5a-2.5\log\left(\frac{b}{b-0.4}\right)-2.5(b-0.4)M_\text{max}
	\end{equation}
	We can take the approximation
	\begin{equation}
		M_\text{tot}\approx 2.5a-2.5\log\left(\frac{b}{b-0.4}\right)\sim \SI{40}{mag}
	\end{equation}
	which is valid because $(b-0.4)\lesssim 1$ and the last term is negligible. Note also $M_\text{max}$ is limited: lower masses produces traces of lower magnitudes, in we go below a given mass threshold the particle is too small to mechanically interact with the atmosphere.
	
	\item At $\rho=\SI{100}{km}$ by definition $M=m$. At this distance, if we neglect projection effects, a unit surface is seen under the angular area
	\begin{equation}
		\Omega = (206265/100)^2 = \SI{4.25e6}{arcsec^2} \sim 0.5\times\ang{0.5}
	\end{equation}
	
	\item Since surface brightness does not depend on the distance (both flux and the solid angle scales with $1/\rho^2$) at any position
	\begin{equation}
		S = M_\text{tot}+2.5\log\Omega = 40+6 = \SI{46}{mag}
	\end{equation}

	
	
\end{enumerate}

\end{document}